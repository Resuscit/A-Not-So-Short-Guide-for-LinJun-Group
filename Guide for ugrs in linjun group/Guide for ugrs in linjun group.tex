%!TEX program = xelatex
% 使用 ctexart 文类,UTF-8 编码
\documentclass{article}
  \usepackage{xecjk,indentfirst}
  \usepackage{amsmath}
  \setlength{\parindent}{2em}
  \setlength{\textheight}{240mm}
  \setlength{\textwidth}{155mm}
  \setlength{\oddsidemargin}{0mm}
  \setlength{\evensidemargin}{0mm}
  \setlength{\topmargin}{-20mm}
  \renewcommand{\baselinestretch}{1.2}
  \title{王林军老师课题组本科生入门指南}
  \author{Chaoqun ZHANG-张超群}
  \date{\today}
  \begin{document}
    \maketitle
    \tableofcontents
    \newpage
    \section{引言}

    \section{Linux基础及服务器使用}
      
    \section{数学基础}
     
    \section{量子力学}

    \section{量子化学-电子结构基础}

    \section{非绝热动力学与势间跳跃方法}
    在Born-Oppenheimer近似下,我们可以针对特定的分子构型(核的位置)计算对应的电子能量,对于不同的核坐标$\mathbf{R}$(习惯上大写
    表示原子核坐标,小写表示电子坐标),可以给出一系列电子能量$U=U(\mathbf{R})$,这就是我们通常所说的势能面。需要注意的是,
    势能面是基于BO近似的结果,这是一切的前提。
    
    (TODO需要修改)BO近似几乎是量子化学中最重要的近似条件,可以适用于我们平常遇到几乎全部静态问题和很多动态问题,BO近似下基态的势能面也是比较合理的。
    但是在一些情况下,尤其是当研究的体系涉及多个激发态——比如光激发然后通过非辐射方式回到基态过程时,BO近似就显示出了其弊端。一类
    常见的BO近似失效的情景称为圆锥交叉(conical intersections),理想的圆锥交叉指的是两个势能面的至少两个维度在某一分子构型上简并(交叉)。
    通过早期实验光谱上和理论上的研究,人们发现涉及圆锥交叉的势能面的动力学是BO近似无法描述的。
    
    为了解决这个问题,必须把原子核和电子的运动一起加进动力学模拟的过程中,这类动力学通常称为非绝热动力学(non-adiabatic coupling)。处理核的运动
    有多种方式,经典的分子动力学就是只有原子核的运动,不属于非绝热动力学的范畴,但是经典运动是处理原子核的一种方式,另外还可以将原子核的运动
    也用量子力学求解,进行全量子的非绝热动力学模拟。王林军老师课题组常用的势间跳跃(surface hopping)方法,就是原子核做经典处理,电子做量子处理
    的一种混合量子经典的非绝热动力学方法。
      \subsection{非绝热耦合}
        作为“动力学”下的讨论,我们一定会用到含时Schr\"odinger方程:
        \begin{equation}
          i \hbar \frac{\mathrm{d} | \psi \rangle}{\mathrm{d} t}=\hat{H} | \psi \rangle
        \end{equation}
        或者是与之等价的量子Liouville方程:
        \begin{equation}
          \frac{\mathrm{d} \hat{\rho}}{\mathrm{d} t}=\frac{-i}{\hbar}[\hat{H}, \hat{\rho}]
        \end{equation}
        Liouville方程的优点是直接演化密度矩阵(TODO电子结构部分解释),动力学所关心的问题就是波函数/密度矩阵如何随时间演化,而在一般的量子力学
        语境中,波函数的演化通常是波函数系数的演化,含时的波函数可以写作不含时基组的线性组合
        \begin{equation}
          |\psi(\mathbf{r}, \mathbf{R}, t)\rangle=\sum_{j} c_{j}(t) |\phi_{j}(\mathbf{r} ; \mathbf{R})\rangle
        \end{equation}
        这个不含时基组可以是(也可以不是)定态Schr\"odinger方程的解,我们讨论一个更一般的情况,这个基组只被要求满足正交归一性,
        带入含时Schr\"odinger方程得到
        \begin{equation}
          \sum_j c_j \hat{H}|\phi_j\rangle=i \hbar \frac{\mathrm{d} | \psi \rangle}{\mathrm{d} t}=i \hbar\sum_j\left(\dot{c}_j|\phi_j\rangle+c_j\frac{\mathrm{d}}{\mathrm{d}\mathbf{R}}|\phi_j\rangle\dot{\mathbf{R}}\right)
        \end{equation}
        在上式左右两端乘以$\langle\phi_i|$就得到了
        \begin{equation}
          i \hbar \dot{c}_{i}=\sum_{j} c_{j}\left(V_{i j}-i \hbar \dot{R} \cdot d_{i j}\right)
          \label{ihcdot}
        \end{equation}
        其中定义了$V_{ij}=\langle\phi_i|\hat{H}|\phi_j\rangle$和$d_{ij}=\langle\phi_i|\frac{\mathrm{d}}{\mathrm{d}\mathbf{R}}|\phi_j\rangle$,
        后者称为非绝热耦合(non-adiabatic coupling,NAC),该项所引发的问题是非绝热动力学的领域核心问题之一。在这里,我们可以通过另一种方式来理解它
        名字的来源以及它的物理意义:如果我们用最简单的数值方法计算非绝热耦合,即
        \begin{equation}
          d_{ij}=\langle\phi_i|\frac{\mathrm{d}}{\mathrm{d}\mathbf{R}}|\phi_j\rangle\approx\langle\phi_i(\mathbf{R})|\frac{|\phi_j(\mathbf{R}+\Delta\mathbf{R})\rangle-|\phi_j(\mathbf{R})\rangle}{\Delta\mathbf{R}}
          =\frac{\langle\phi_i(\mathbf{R})|\phi_j(\mathbf{R}+\Delta\mathbf{R})\rangle}{\Delta\mathbf{R}}
        \end{equation}
        上式的最后结果说明非绝热耦合直接正比于不同核位置的波函数之间重叠,要知道如果这些态在同一个
        核坐标下,他们满足正交性。之所以有不等于0的非绝热耦合,就是因为在不同核坐标下的不同电子态间有“耦合”,
        而这个耦合正是超越BO近似所引入的,是“非绝热”而引入的。
        
        最后,尽管计算非绝热耦合在很多时候都非常重要,我们在这篇向导中不作特别仔细的讨论,通常有类似上式的数值求解方法
        和通过解析导数的方法求解,但是具体要在不同表象下讨论,下一小节我们就要介绍所谓绝热表象与透热表象。
      \subsection{绝热表象与透热表象}
        本文中所说的绝热(adiabatic)与透热(diabatic)与热学中相关概念并无直接联系,属于量化领域常用的一种表达。
        绝热是基于Born–Oppenheimer近似而言的,需要注意由于翻译的问题,请不要混淆非绝热(non-adiabatic)和透热(diabatic)
        两个概念。

        根据Troy Van Voorhis在他那篇著名的综述(Annu. Rev. Phys. Chem. 2010. 61:149–70)中的说法,绝热态定义
        为BO近似下的哈密顿量的本征态,也就是说在BO近似下解定态Schr\"odinger方程(电子结构计算)得到的波函数就是
        绝热态,即$\hat{H}|\phi_i\rangle=E_i|\phi_i\rangle$。而透热态是指不随核坐标(分子构型)变化的态,最常用的
        就是Troy提到的NaCl的例子,透热态其实是量子化学家们为了理解图像和计算的方便提出的不真实存在的量子态,如不管
        Na和Cl的距离多大,永远为两个中性原子的态,或者是永远是两个离子的态。这样说来十分抽象,但是本质上就是
        其波函数不随核坐标变化而变化,一个严格的定义为透热态的非绝热耦合为0(甚至很多时候简化为$\frac{\mathrm{d}}{\mathrm{d}\mathbf{R}}|\phi_j\rangle=0$)。
        在电荷转移中的图像更加清晰,如果电荷转移发生在两个分子间,透热态通常定为电荷局域在第一个分子上的态和局域在第二个分子上的态,
        而真实的态(电荷分散在两个分子上,也就是绝热态)是两个透热态的线性组合。

        在实际操作中,绝热态是很容易得到的,因为绝热态是哈密顿量本征态,我们可以用各种电子结构的计算方法得到他们,但是
        在绝热表象下,两个绝热态之间存在非绝热耦合,势能面间也会有圆锥交叉或者trivial crossing,这几项往往会给动力学计算带来
        很大难度,因此人们很多时候会希望通过绝热态的线性组合来构建透热态,在透热表象下进行动力学模拟,这个过程称为透热化(diabatization)。透热化的
        方法非常多,本向导中也不作仔细的展开,但是根据Troy的综述,真正的透热态是难以通过绝热态线性组合得到的,所以很多文章中会使用
        构建半透热态(quasi-diabatic state)的说法。需要注意,根据非绝热耦合等于零的定义,透热态按照某个固定的系数的线性组合还是透热态,也就是
        透热态并不是唯一的,这种不唯一性给了量子化学家构建透热态的更大的自由度。这也是目前透热化的方法多种多样的一个原因。

        在透热表象下,由于透热态不是哈密顿量的本征态,所以此时哈密顿量不再是对角矩阵,而是有非零的非对角元,在某些问题(如电荷转移)中,
        这些非对角元通常被叫做转移积分(transfer integrals),对角化透热态哈密顿量应该能重新得到绝热态的能量。
        在透热表象下,可以认为式(\ref{ihcdot})中的$V_{ij}$就是转移积分,所以在某种程度上说,即使透热化并不完全,仍有很小数值的
        非绝热耦合,只要该项对转移积分是一个小量,对动力学的影响也可以忽略,但是在绝热表象中由于$V_{ij}=0$,因此
        非绝热耦合即使有些时候较小,仍是十分重要的。
      \subsection{势间跳跃方法与FSSH}
  \end{document} 